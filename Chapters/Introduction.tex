\chapter{Formatting - TODO}

\begin{itemize}
  \item fix acronym capitalization
  \item maybe have acronyms rendered in italic
  \item section references in caps?
\end{itemize}
\chapter{Introduction}\label{chap:introduction}

Many decision making problems are subject to inherent uncertainty. Examples of
such problems range from aircraft collision avoidance and robotic navigation tasks to
to applications in health care and medical treatment
\cite{kochenderfer2012next, schaefer2005modeling}.\todo{cite} While humans have
developed good intuition for decision making problems present in their day to
day lives, many of the same tasks -- like planning in autonomous driving --
pose difficult problems for robotic agents \cite{levinson2011towards}.

Actively considering uncertainty in planning promises to improve robustness,
safety and performance of the system. In conventional control theory a typical
approach is to model uncertainty in terms of a bounded disturbance,
robustifying the controller through reasoning over worst case disturbance
sequences. However, modelling the disturbance as an adversarial player is often
impractical, since long tail distributions may cause the controller to come up
with overly conservative, thus poorly performing plans.\todo{cite}

Therefore, a tremendous amount of research has focused on incorporating
uncertainty in decision making through probabilistic models, rather than
adversarial game type approaches \cite{roy1999coastal, amato2015planning,
fisac2018probabilistically, choudhury2019dynamic}.

One of the most general framework for modeling uncertainty in sequential
decision making in a probabilistic fashion is provided by the \ac{pomdp}.
Formulating and solving a planning problem as a \ac{pomdp} allows the planner
to reason over both, state and outcome uncertainty. Also, by taking into
account future observations, solving a \ac{pomdp} gives rise to behavior
through computation that actively performs information gathering.

While \acp{pomdp} provide a comprehensive way of taking into account
uncertainty in the planning procedure, finding the optimal solution to these
problems is in practice often intractable since in worst case it can not be
done in polynomial time \cite{papadimitriou1987complexity}. Therefore, in
robotics applications, characterized by limited compute and real time
constraints, \ac{pomdp} solution methods are traditionally avoided. Instead,
simplifications are made that neglect the partial observability or make other
assumptions about the problem structure \cite{sadigh2016information,
fisac2018probabilistically}.\todo{still not happy with this paragraph}

\section{Problem Statement}

\todo[inline]{Motivated by recent progress in research and tooling (Julia,
Solvers etc.) we do a feasibility study.}

This work aims to provide insight into the use of \acp{pomdp} for modelling and
solving planning problems in robotics. This results in robust policies for
optimized interaction with uncertain environments. We show how this methodology
provides solutions to problems where uncertainty otherwise, either keeps
engineers from solving them in a principled manner or forces them to make
simplifications that compromise performance and robustness.


We look at two application domains where uncertainty affects decision making in
different ways. First, we focus on simultaneous localization and planning,
a problem characterized by inherent uncertainty in the physical state of the
robot. Subsequently, we examine motion planning in a shared space with a human
agent where uncertainty forces the robot to reason over the latent human
behavior model. We compare the performance of our provided solution to a domain
specific approximation provided by \cite{fisac2018probabilistically} and
discuss the mechanisms through which \acp{pomdp} improve the agent performance
in this domain.

\begin{itemize}
  \item we discuss the mechanisms which make considering uncertainty important
  \item \dots and discuss the differences between the two problems.
\end{itemize}

\section{Outline}
\todo[inline]{write, when structure has converged}

\begin{description}
  \item[Theory] Theoretical properties and background of POMDPs. Solution methods: POMCPOW, DESPOT.
  \item[Experiments] Applications of POMDPs to two problem domains: Simultaneous localization an planning, Motion planning with latent human intentions.
  \item[Summary] The summary of this work.
\end{description}
