\chapter{Theory}

% This is some test text.\todo{Test}

\todo[inline]{brief in-chapter outline as soon as this has converged.}
\todo[inline]{point to the fact that only a brief overview can be given and that fundamentals
              (all the way up to MDPs) are assumed to be known by reader and refer to Mykel and other sources for further information}

\section{Notation}

\todo[inline]{Formalize most important notation and point to "Decision making
under uncertainty" as source for this notation}

\section{Partially Observable Markov Decision Processes}

\todo[inline]{formal definition of a POMDP as derived from a MDP}

\missingfigure{Dynamic Bayes Net for POMDP}

\todo[inline]{brief definition of solutions to POMDPs as "conditional plans" and point out that in general
the full solution is intractable. Point to complexity theory with some good source for this. (e.g. Mykel)}

\todo[inline]{Theoretical properties by modelling things in such a way. What can be gained? What is the difference?}

\section{Online POMDP Solvers}

\todo[inline]{The solvers described here were chosen as they are state of the art and show good perfomance over a range of problems. Point to Zach's paper for solver perfomance comparison and DESPOT paper.}

\section{POMCPOW}

\todo[inline]{write:}
\begin{itemize}
  \item explain the basic idea of POMCPOW as Monte Carlo with DPW.
  \item is extension of POMCP with weighted particle beliefs.
\end{itemize}
\missingfigure{Pseudo Code with formal description in text.}

\missingfigure{graphical model of POMCP-Tree with weighted scenarios.}

\section{DESPOT (maybe drop and try to solve planning+localization with POMCPOW instead?)}

\todo[inline]{write:}
\begin{itemize}
  \item formal, high-level idea of a DESPOT (idea of scenarios etc.)
  \item search on a DESPOT with bounds (and outline of the algorithm)
  \item point to literature for convergence guarantees
\end{itemize}

\missingfigure{DESPOT tree visualization}
\missingfigure{DESPOT algorithm Pseudo Code}
