\iffalse
\chapter{Discussion}\label{chap:discussion}

\todo[inline]{Note: The analysis/discussion for each application example has
been "inlined" with the qualitative analysis at the end of each chapter. In
this chapter I now only summarize the mechanisms and give some advice for "when
does it make sense to model a problem as a POMDP". Maybe this is not enough for
a whole chapter and rather redundant.}

The experiments presented in \cref{chap:localization-and-planning,chap:hri}
show that \ac{pomdp} solvers are able to generate well performing behaviors for
different planning problems. The quantitative and qualitative analysis
performed for these examples reveal that the benefits provided by this kind of
reasoning are cause by different mechanisms.

\begin{itemize}
  \item prob obstacles is technically a nice idea but it only works for this very problem structure
  \item this kind of reasoning does not work when the human reacts to the robot since open loop predictions
  require knowledge of the robot position in the future.
\end{itemize}

\todo[inline]{This section will mainly contain the "lessons learned"
+ a comparison between the problems, explaining the mechanisms which lead to
the improvement in each case.}

\begin{itemize}
  \item \acp{pomdp} provide an elegant way of closing the loop between
    prediction (and perception models) and planning.
  \item This provides benefits through various mechanisms:
  \item Simultaneous localization and planning: Using a \ac{pomdp} leads to active
    information gathering and provides a principled, optimization based
    approach to solving this problem (even under massive uncertainty). Solving
    it without such procedure is not straightforward, since planning on expectation
    with these highly multi-modal belief topologies is impractical.
  \item Planning with latent human intentions: The \ac{pomdp} based solution
    allows the robot to consider future observations, making the robots plans
    less conservative because the agent knows that uncertainty about the humans intentions
    will be reduced in the future. For this very problem structure, neglecting future observations
    provides a principled way of solving this problem. Through this
    approximation, the problem is simplified to a problem that is still
    NP-hard, but well studied. Good heuristics exist, such that search is well guided to find results
    withing reasonable planning horizons. However, sacrificing performance (in
    particular for models with unbounded uncertainty like constant velocity).
    \begin{itemize}
      \item Future work: \ac{pomdp} solution method is safer than planning with
      probabilistic obstacles while reaching the goal in fewer number of steps.
      However, probabilistic obstacles can provide a-priory estimate of safety
      while for \acp{pomdp} this can only be shown empirically through large
      scale simulations. It would be nice to provide a safety assurance for
      this special type of \acp{pomdp} (Mixed Observability Markov Decision
      Process where the safety only depends on some unactuated decoupled part
      of the state.)
    \end{itemize}
\end{itemize}
\fi
