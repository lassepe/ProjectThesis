\chapter{Summary}

This work examines the use of \acf{pomdp} for planning in uncertain
environments in the context of robotics. At the example of two continuous-space
application domains we provide a detailed comparison of approximate numerical
\ac{pomdp} solutions to approaches based on problem specific simplifications.

We begin by reviewing some of the fundamentals of sequential decision making
under uncertainty. We introduce the theoretical framework of \acp{pomdp} and
present two state-of-the-art approximate online \ac{pomdp} solvers:
\acf{despot}, and \acf{pomcpow}. Furthermore, we briefly discuss our choice of
tools and software framework that is suitable for modelling and solving
\acp{pomdp}.

Based on this theoretical introduction we examine use of \acp{pomdp} for
robotic planning problems under uncertainty. Specifically, we make the
following contributions.\\
\todo[inline]{Add results of evaluation to each contribution.}
First, we systematically quantify the performance
advantage of full \ac{pomdp} solutions with both analytic and empirical
heuristic guidance compared to problem specific simplifications in a
continuous-space problem domains. For this purpose we focus on
\emph{simultaneous localization and planning}, a problem characterized by
inherent uncertainty in the physical state of the robot.\\
The second contribution is a detailed comparison of the problem specific
planning approach by \cite{fisac2018probabilistically} to a state-of-the-art
\ac{pomdp} solver. This comparison aims to show how well a problem may be
solved without considering the full complexity of \acp{pomdp} and to which
extent such an approximation can not keep up with the full \ac{pomdp} solution.
For this purpose we study an instance of motion planning in a shared space with
a human actor. In this application domain, latent human intentions force the
agent to consider uncertainty in order to reach its goal safely.\\
Finally, as part of our work on motion planning with latent human intentions
we contribute an implementation of a software model to simulate
interacting humans and robots with near real time \ac{pomdp} planning
capabilities. This implementation is designed to accommodate convenient
interchangeability of different human models and builds upon \pomdpsjl to allow
for a direct comparison of different planning approaches.


\todo[inline]{Write once everything else has converged.}

% This results in robust policies for optimized interaction with uncertain
% environments. We show how this methodology provides solutions to problems where
% uncertainty otherwise, either keeps engineers from solving them in a principled
% manner or forces them to make simplifications that compromise performance and
% robustness.

Mention contribution again.

\chapter{Formatting - TODO}

\begin{itemize}
  \item genitive "s" (robots vs. robot's, its vs it's) \todo{"its" is analogous to "his" or "hers"; it's is a contraction. You'll probably never use it's in technical writing.}
  \item Introduction must be page one (see TOC)
  \item (done) ix acronym capitalization
  \item maybe have acronyms rendered in italic
  \item section references in caps?
  \item look for common spelling mistakes
  \item look for "will", most things should be "present"
  \item avoid "don't", "won't", "doesn't" etc.\
  \item flows $\to$ follows
  \item make sure that there is only one version of writing "Monte-Carlo" (hyphen)
  \item policy-tree vs belief-tree
  \item maybe avoid using "That is," at the beginning of a sentence but it is not that important
\end{itemize}
