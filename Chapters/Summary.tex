\chapter{Summary}\label{chap:summary}

This work examines the use of \acfp{pomdp} for planning in uncertain
environments in the context of robotics. At the example of two continuous-space
application domains we provide a detailed comparison of approximate numerical
\ac{pomdp} solutions to approaches based on problem specific simplifications.

We begin by reviewing some of the fundamentals of sequential decision making
under uncertainty. We introduce the theoretical framework of \acp{pomdp} and
present two state-of-the-art approximate online \ac{pomdp} solvers:
\acf{despot}, and \acf{pomcpow}. Furthermore, we briefly discuss our choice of
tools and software framework for modelling and solving
\acp{pomdp}.\\
Based on this theoretical introduction we examine the use of \acp{pomdp} for
robotic planning problems under uncertainty at the example of two problem
domains.\\
First, we study an instance of \emph{simultaneous localization and planning}.
For this application domain we systematically quantify the performance
advantage of \ac{despot} and \ac{pomcpow} with both analytic and empirical
heuristic guidance compared to two problem specific baselines. Our
results show that by integrating domain knowledge in a computationally
efficient manner, behaviors generated by full \ac{pomdp} approaches are significantly
safer and

\todo[inline]{add results}
Subsequently, we study an instance of motion planning in a shared space with
a human whose intentions are unknown to the robot. For this purpose we implement a
software model to simulate interacting humans and robots with near real time
\ac{pomdp} planning capabilities. Using this framework, we make a detailed
comparison of \acf{psrp} proposed by \cite{fisac2018probabilistically} to
a \ac{pomcpow} approach.
Our results show that ... \todo[inline]{add results}

% This comparison aims to show how well a problem may be solved without
% considering the full complexity of \acp{pomdp} and to which extent such an
% approximation can not keep up with the full \ac{pomdp} solution.

% This results in robust policies for optimized interaction with uncertain
% environments. We show how this methodology provides solutions to problems where
% uncertainty otherwise, either keeps engineers from solving them in a principled
% manner or forces them to make simplifications that compromise performance and
% robustness.

\chapter{Formatting - TODO}

\begin{itemize}
  \item genitive "s" (robots vs. robot's, its vs it's) \todo{"its" is analogous to "his" or "hers"; it's is a contraction. You'll probably never use it's in technical writing.}
  \item Introduction must be page one (see TOC)
  \item (done) ix acronym capitalization
  \item maybe have acronyms rendered in italic
  \item section references in caps?
  \item look for common spelling mistakes
  \item look for "will", most things should be "present"
  \item avoid "don't", "won't", "doesn't" etc.\
  \item flows $\to$ follows
  \item make sure that there is only one version of writing "Monte-Carlo" (hyphen)
  \item policy-tree vs belief-tree
  \item maybe avoid using "That is," at the beginning of a sentence but it is not that important
\end{itemize}
