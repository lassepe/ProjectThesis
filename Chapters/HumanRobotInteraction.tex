\chapter{Motion Planning with Latent Human Intentions}\label{chap:hri}

\begin{itemize}
  \item robots need to serve the human: autonomous agents usually share the same space as humans
  \item navigation around humans is hard because humans are hard to predict
  \item model mismatch inevitable: humans might not always behave like expected
  \item robot can not model every aspect of the world
  \item with missing features: from a robot perspective the human often acts irrational
  \item deterministic models fail to account for these things
  \item better: stochastic models
  \item two ways in which stochasticity can be helpful
  \begin{itemize}
    \item high level human intention (e.g. where does the human want to walk)
    \item stochastic low level behavior (e.g. how does the human walk)
  \end{itemize}
  \item There has been a lot of work on this and recent work suggests that modelling the human in a stochastic way is helpful to increase robustness. \cite{fisac2018probabilistically}
  \item Fisac simplified the problem by neglecting branching on future observations. Hypothesis: makes the solution too conservative due to growing uncertainty in the future.
  \item This chapter will propose a full POMDP solution (with observation branching) in an effort to investigate the performance gain.
  \item (motivated by the fact that the we have a fast programming language that makes this feasible)
\end{itemize}

\section{Problem Statement}\label{sec:hri-problem-statement}
\section{POMDP Formalization}\label{sec:hri-pomdp-formalization}
\section{Solution Strategies}\label{sec:hri-solutions}
\subsection{Base Line}\label{sec:hri-base-line}
\subsection{POMDP Solution}\label{sec:hri-planners}
\section{Evaluation}\label{sec:hri-evaluation}
