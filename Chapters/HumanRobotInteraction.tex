\chapter{Motion Planning with Latent Human Intentions}\label{chap:hri}

Robots that are designed to assist humans almost inevitably have to operate in
a shared environment. Close \ac{hri} requires autonomous agents to navigate
safely in domains that typically don't feature safety barriers to physically
separate them from humans. Therefore, robots must ensure human safety through
careful planning and robust behaviors. At the same time, trajectories of humans
are hard to predict as they follow complex behaviors whose dynamics are only
partially understood.\todo{cite}. For this reason, research in the past
has moved from simple rule-based and deterministic models \todo{social forces
etc.} towards data driven probabilistic predictions that approximate the future
as a distribution over trajectories \todo{cite: social lstm, and learning to
predict trajectories}.

Incorporating stochasticity in the prediction pipeline allows the planner to
model uncertainty over both, high-level intentions (e.g. \emph{where} does the
human want to go), and low-level motion behavior (e.g. \emph{how} does the
human want to go there). Planning strategies that take this uncertainty into
account promise to provide robustified and potentially more efficient policies
for navigation around humans. While the properties arising from this kind of
reasoning are desirable for many \ac{hri} problems, reasoning over
distributions of possible futures may pose a challenging \ac{pomdp}. Therefore,
a lot of research has focused on recovering some of these properties by
proposing domain specific simplifications for these applications
\cite{fern2007decision, sadigh2016information, javdani2018shared, fisac2018probabilistically}.

\begin{itemize}
  \item robots need to serve the human: autonomous agents usually share the same space as humans
  \item navigation around humans is hard because humans are hard to predict
  \item model mismatch inevitable: humans might not always behave like expected
  \item robot can not model every aspect of the world
  \item with missing features: from a robot perspective the human often acts irrational
  \item deterministic models fail to account for these things
  \item better: stochastic models
  \item two ways in which stochasticity can be helpful
  \begin{itemize}
    \item high level human intention (e.g. where does the human want to walk)
    \item stochastic low level behavior (e.g. how does the human walk)
  \end{itemize}
  \item There has been a lot of work on this and recent work suggests that modelling the human in a stochastic way is helpful to increase robustness. \cite{fisac2018probabilistically}
  \item Fisac simplified the problem by neglecting branching on future observations. Hypothesis: makes the solution too conservative due to growing uncertainty in the future.
  \item This chapter will propose a full POMDP solution (with observation branching) in an effort to investigate the performance gain.
  \item (motivated by the fact that the we have a fast programming language that makes this feasible)
\end{itemize}

\section{Problem Statement}\label{sec:hri-problem-statement}
\section{POMDP Formalization}\label{sec:hri-pomdp-formalization}
\section{Solution Strategies}\label{sec:hri-solutions}
\subsection{Base Line}\label{sec:hri-base-line}
\subsection{POMDP Solution}\label{sec:hri-planners}
\section{Evaluation}\label{sec:hri-evaluation}
